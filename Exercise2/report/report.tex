% !TeX spellcheck = en_GB
% Die erste (unkommentierte) Zeile im Dokument legt immer die
% Dokumentklasse fest
\documentclass{scrartcl} 

% Präambel:
% Einbinen von zusätzlichen Paketen. Falls für eine Datei keine Endung
% explizit angegeben wird, benutzt LaTeX '.tex'. Im Folgenden wird
% also die Datei 'edv_pakete.tex' eingebunden.
\input{edv_pakete}


% Verzeichnisse mit Abbildungen; kann gestrichen werden,
% falls Sie dies schon in edv_pakete.tex definiert haben:
%\graphicspath{{../report}}

\addbibresource{refs.bib} %Hinzufügen einer Literaturdatenbank aus dem angegebenen Verzeichnis

% Titel, Autor und Datum
\title{Computational Physics}
\subtitle{Exercise 2}
\date{\today}
\author{Christiane Groß, Nico Dichter}

% Jetzt startet das eigentliche Dokument
\begin{document}
	\maketitle
\section{The 2D-Ising-model}
Hamiltonian, expected values for magnetization and epsilon

\section{Deliberations}
\subsection{Numerical cost of calculation}
\paragraph{Energy} Have to iterate over all lattice points: effort proportional lambda.

\paragraph{change in energy for one spin flip}
naive: calculate two Hamiltonians, subtract: scales with lambda. 
Possible: only use nearest neighbours, as shown in lecture for 1D: effort only depends ion dimension, not lattice size.

\subsection{Meaning of J} critical J: phase transition from randomly ordered to ordered.

\subsection{determining errors}

\section{simulation}
describe energy\_change, matrices, thermalising, sweeps

\section{Results}

\subsection{$\langle m\rangle$ for fixed $J$ and varying $h$}
plot, description

\subsection{$\langle \epsilon\rangle$ for fixed $h$ and varying $J$}
plot, description, comparison literature

\subsection{$\langle |m|\rangle$ for fixed $h$ and varying $J$}
plot, description, comparison literature
see sharp drop, phase transition
\paragraph{$\langle |m|\rangle$ vs. $\langle m\rangle$}
For $h=0$: everything cancels out, m=0 everywhere because simulation randomly flops between positive and negative behaviour, even at ordered phase due to finite length.
Use absolute value of m so it can't cancel out.

\paragraph{thermodynamical limit}



\newpage	
\listoffigures
\printbibliography
\end{document}