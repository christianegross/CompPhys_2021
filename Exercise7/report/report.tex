%!TeX spellcheck = en_GB
% Die erste (unkommentierte) Zeile im Dokument legt immer die
% Dokumentklasse fest
\documentclass{scrartcl} 

% Präambel:
% Einbinen von zusätzlichen Paketen. Falls für eine Datei keine Endung
% explizit angegeben wird, benutzt LaTeX '.tex'. Im Folgenden wird
% also die Datei 'edv_pakete.tex' eingebunden.
\input{edv_pakete}


% Verzeichnisse mit Abbildungen; kann gestrichen werden,
% falls Sie dies schon in edv_pakete.tex definiert haben:
%\graphicspath{{../report}}

\addbibresource{refs.bib} %Hinzufügen einer Literaturdatenbank aus dem angegebenen Verzeichnis

% Titel, Autor und Datum
\title{Computational Physics}
\subtitle{Exercise 7}
\date{\today}
\author{Christiane Groß, Nico Dichter}

% Jetzt startet das eigentliche Dokument
\begin{document}
	\maketitle
	
\section{Theory}

\subsection{Discretization of the Lippmann-Schwinger-equation}

Using some definitions given in the lectures and on the sheet, we are able to discretize the Lippmann-Schwinger-equation given on the sheet. we use Gauss-Legendre-integration and get the following matrix-equation:

\begin{longtable}{>{$\displaystyle}r<{$}>{$\displaystyle}c<{$}>{$\displaystyle}l<{$}}
	f(p'')&=&2\mu p''^2V_l(p, p'')t_l(p'', p')\\
	q&=&p_N\\
		
	V(p_i, p_j)&=&V_{ij}\\
		
		
	t(p_i, p_j)&=&t_{ij}\\
		
	\int_{0}^{\infty}\dd p'' g(p'')&=&\sum_{k=0}^{N-1}\omega_k g(p''_k)\\
	
	&&\\
	
	\dashint_{0}^{p_{max}}\dd p''\frac{1}{q^2-p''^2}&=&\lim\limits_{\epsilon\to 0}\int_{0}^{q-\epsilon}\dd p''\frac{1}{(q+p'')(q-p'')}+\int_{q+\epsilon}^{p_{max}}\dd p''\frac{1}{(q+p'')(q-p'')}\\
	&=&\lim\limits_{\epsilon\to 0}\frac{1}{2q}\left(\int_{0}^{q-\epsilon}\dd p''\frac{1}{q+p''}-\frac{1}{p''-q}+\int_{q+\epsilon}^{p_{max}}\dd p''\frac{1}{q+p''}-\frac{1}{p''-q}\right)\\
	&=&\lim\limits_{\epsilon\to 0}\frac{1}{2q}\left(\ln\left(\frac{q+p''}{p''-q}\right)_0^{q-\epsilon}+\ln\left(\frac{q+p''}{p''-q}\right)_{q+\epsilon}^{p_{max}}\right)\\
	&=&\frac{1}{2q}\ln\left(\frac{p_{max}+q}{p_{max}-q}\right)\\
	
	&&\\

	
	V_l(p, p')&=&t_l(p, p')-\int_0^{\infty}\dd p''\frac{f(p'')-f(q)}{q^2-p''^2}-f(q)	\dashint_{0}^{p_{max}}\dd p''\frac{1}{q^2-p''^2}+i\pi\frac{f(q)}{2q}\\
	
	V_l(p, p')&=&t_l(p, p')-2\mu\int_0^{\infty}\dd p''\frac{p''^2V_l(p, p'')t_l(p'', p')-q^2V_l(p, q)t_l(q, p')}{q^2-p''^2}\\
	&&-2\mu q^2V_l(p, q)t_l(q, p')\frac{1}{2q}\ln\left(\frac{p_{max}+q}{p_{max}-q}\right)+i\pi\frac{2\mu q^2V_l(p, q)t_l(q, p')}{2q}\\
	
	V_l(p, p')&=&t_l(p, p')-2\mu\int_0^{\infty}\dd p''\frac{p''^2V_l(p, p'')t_l(p'', p')-q^2V_l(p, q)t_l(q, p')}{q^2-p''^2}\\
	&&-\mu qV_l(p, q)t_l(q, p')\ln\left(\frac{p_{max}+q}{p_{max}-q}\right)+i\pi\mu qV_l(p, q)t_l(q, p')\\
	
	V_{ij}&=&t_{ij}-2\mu\sum_{k=0}^{N-1}\omega_k\left(\frac{p_k^2V_{ik}t_{kj}}{q^2-p_k^2}-\frac{q^2V_{iN}t_{Nj}}{q^2-p_k^2}\right)-\mu qV_{iN}t_{Nj}\ln\left(\frac{p_{max}+q}{p_{max}-q}\right)+i\pi\mu qV_{iN}t_{Nj}\\
	
	&=&\sum_{k=0}^{N-1}\left(\delta_{ik}-\frac{2\mu p_k^2V_{ik}}{q^2-p_k^2}\omega_k\right)t_{kj}\\
	&&+\left(\delta_{iN}+\sum_{k=0}^{N-1}\frac{2\mu q^2V_{iN}}{q^2-p_k^2}\omega_k-\mu qV_{iN}\ln\left(\frac{p_{max}+q}{p_{max}-q}\right)+i\pi\mu qV_{iN}\right)t_{Nj}\\
	&=&\sum_{k=0}^{N}A_{ik}t_{kj}\\
	
	
	\end{longtable}
	
	This is exactly the matrix given on the sheet.
	
	The potential was given in the lectures as \[
	V_l(p, p')=2\pi\int_{-1}^{1}\dd x P_l(x)\frac{1}{p^2+p'^2-2pp'x+\mu^2}
	\]
	
	We also discretize this integral using the Gauss-Legendre-approximation. 
	
	For determining $t(q, q)=t_{NN}$, we use the following manipulation:
	\[
	V=At \Leftrightarrow t=A^{-1}V \Leftrightarrow t_{NN}=(A^{-1}V)_{NN}
	\]
	
\section{Implementation}

Our code is in the github-repo \url{https://github.com/christianegross/CompPhys\_2021}. The simulation itself is in Exercise7/src/main/main.c, the gnuplotscript is in Exercise7/report/plot.gp. 
\newpage
\listoffigures
\listoftables
\printbibliography
\end{document}
