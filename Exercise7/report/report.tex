%!TeX spellcheck = en_GB
% Die erste (unkommentierte) Zeile im Dokument legt immer die
% Dokumentklasse fest
\documentclass{scrartcl} 

% Präambel:
% Einbinen von zusätzlichen Paketen. Falls für eine Datei keine Endung
% explizit angegeben wird, benutzt LaTeX '.tex'. Im Folgenden wird
% also die Datei 'edv_pakete.tex' eingebunden.
% Die erste Zeile im Dokument legt immer die Dokumentklasse fest
%\documentclass[notitlepage]{scrreprt}
    % Die wichtigsten Dokumentklassen:
    %   scrbook, scrreprt, scrartcl, beamer, standalone
    % Einige gängige Optionen für \documentclass:
    %   ngerman
    %   titlepage, notitlepage
    %   onecolumn, twocolumn
    %   oneside, twoside
    %wird in Hauptdatei festgelegt

% Präambel

% Einige KOMA-Script-Optionen
\KOMAoptions{fontsize=12pt,paper=a4}      %Schriftgröße, Papierformat
\KOMAoptions{DIV=11}                      % Parameter mit dem man den Seitenrand ändern kann
\KOMAoptions{listof=totoc}

% Hier werden einige Pakete eingebunden
\usepackage[utf8]{inputenc}               % Direkte Eingabe von ä usw. Input=Eingabe
\usepackage[T1]{fontenc}                  % Font Kodierung für die Ausgabe Font=Ausgabe
\usepackage[english]{babel}               % Verschiedenste sprach-spezifische Extras, ngerman für neue deutsche Rechtschreibung, auch UK oder US möglich
\usepackage[autostyle=true]{csquotes}     % Intelligente Anführungszeichen, arbeitet mit Babel zusammen
%

\usepackage{amsmath}%Mathedarstellung
\usepackage{commath}%Mathedarstellung
\usepackage{physics}%Physik-Symbole
%\usepackage{IEEEtrantools}%IEEEeqnarray
%
\usepackage{siunitx}   % Intelligentes Setzen von Zahlen und Einheiten
%\sisetup{locale = DE}  % Deutsch als locale für die Zahlen und Einheiten
%http://tex.stackexchange.com/questions/2291/how-do-i-change-the-enumerate-list-format-to-use-letters-instead-of-the-defaul

\usepackage{enumitem}%erlaubt u.A. die Aufzählung mit Buchstaben, gefunden auf http://tex.stackexchange.com/questions/2291/how-do-i-change-the-enumerate-list-format-to-use-letters-instead-of-the-defaul
%
\usepackage[varg]{txfonts}                % Schönere Schriftart, muss nach amsmath, damit keine Fehlermeldung kommt
\usepackage{graphicx} %einbinden von Figuren/Bildern
\graphicspath{{figs/}} % Stammverzeichnis der verwendeten Bilder, muss im selben Ordner wie Hauptdatei sein
%
\usepackage[backend=biber, style=numeric, sorting=none]{biblatex}
%Verwenden von \cite in \footnote: Bibliographie drucken lassen, mehrmals kompilieren
\usepackage{hyperref}%erzeugt klickbare Elemente
\usepackage[all]{hypcap}%hyperref-befehle springen zum oberen Rand des Bildes
% Zum Einbinden von Programmcode verwenden wir das listings-Paket
\usepackage{listings}

% Für Syntax-Highlighting:
\usepackage{xcolor}

\usepackage{longtable}

% Die folgenden listings-Einstellungen sind nötig, um
% deutsche Umlaute und die Tilde (~) in listings-Umgebungen
% verwenden zu können.
\lstset{
    basicstyle=\ttfamily,    
    literate={~} {$\sim$}{1} % set tilde as a literal
    {ö}{{\"o}}1
    {ä}{{\"a}}1
    {ü}{{\"u}}1
    {ß}{{\ss}}1
    {Ö}{{\"O}}1
    {Ä}{{\"A}}1
    {Ü}{{\"U}}1
}

% Farben für Code-Syntaxhighlighting und Weiteres festlegen:
\lstset{
    % Keine besondere Markierung für Leerzeichen in Codes
    showspaces=false,               
    showstringspaces=false,         
    % Farebn für Code-Kommentare und Schlüsselworte:
    commentstyle=\color{red},       % comment style
    keywordstyle=\color{blue},      % keyword style
    stringstyle=\color{orange},		% string style
    breaklines=true,
    numbers=left,                    % where to put the line-numbers; possible values are (none, left, right)
    numbersep=5pt,                   % how far the line-numbers are from the code
    stepnumber=5, 					%how often there are line numbers in code listings
    tabsize=4, 						%default tabsize set to 4 spaces
    %language=python,
    }
%gefunden auf https://en.wikibooks.org/wiki/LaTeX/Source_Code_Listings
%eigene Kommandos/Abürzungen
\newcommand{\tb}{\textbackslash}
\newcommand{\txt}{\texttt}
\newcommand{\umt}{u_{(i+i\%2)/2}^{(2a)}}
\newcommand{\utmt}{u_{(i-2+i\%2)/2}^{(2a)}}
\newcommand{\uti}{\tilde{u}_i^{(a)}}
\newcommand{\utio}{\tilde{u}_{i-1}^{(a)}}




% Verzeichnisse mit Abbildungen; kann gestrichen werden,
% falls Sie dies schon in edv_pakete.tex definiert haben:
%\graphicspath{{../report}}

\addbibresource{refs.bib} %Hinzufügen einer Literaturdatenbank aus dem angegebenen Verzeichnis

% Titel, Autor und Datum
\title{Computational Physics}
\subtitle{Exercise 7}
\date{\today}
\author{Christiane Groß, Nico Dichter}

% Jetzt startet das eigentliche Dokument
\begin{document}
	\maketitle
	
\section{Theory}

\subsection{Discretization of the Lippmann-Schwinger-equation}

Using some definitions given in the lectures and on the sheet, we are able to discretize the Lippmann-Schwinger-equation given on the sheet. we use Gauss-Legendre-integration and get the following matrix-equation:

\begin{longtable}{>{$\displaystyle}r<{$}>{$\displaystyle}c<{$}>{$\displaystyle}l<{$}}
	f(p'')&=&2\mu p''^2V_l(p, p'')t_l(p'', p')\\
	q&=&p_N\\
		
	V(p_i, p_j)&=&V_{ij}\\
		
		
	t(p_i, p_j)&=&t_{ij}\\
		
	\int_{0}^{\infty}\dd p'' g(p'')&=&\sum_{k=0}^{N-1}\omega_k g(p''_k)\\
	
	&&\\
	
	\dashint_{0}^{p_{max}}\dd p''\frac{1}{q^2-p''^2}&=&\lim\limits_{\epsilon\to 0}\int_{0}^{q-\epsilon}\dd p''\frac{1}{(q+p'')(q-p'')}+\int_{q+\epsilon}^{p_{max}}\dd p''\frac{1}{(q+p'')(q-p'')}\\
	&=&\lim\limits_{\epsilon\to 0}\frac{1}{2q}\left(\int_{0}^{q-\epsilon}\dd p''\frac{1}{q+p''}-\frac{1}{p''-q}+\int_{q+\epsilon}^{p_{max}}\dd p''\frac{1}{q+p''}-\frac{1}{p''-q}\right)\\
	&=&\lim\limits_{\epsilon\to 0}\frac{1}{2q}\left(\ln\left(\frac{q+p''}{p''-q}\right)_0^{q-\epsilon}+\ln\left(\frac{q+p''}{p''-q}\right)_{q+\epsilon}^{p_{max}}\right)\\
	&=&\frac{1}{2q}\ln\left(\frac{p_{max}+q}{p_{max}-q}\right)\\
	
	&&\\

	
	V_l(p, p')&=&t_l(p, p')-\int_0^{\infty}\dd p''\frac{f(p'')-f(q)}{q^2-p''^2}-f(q)	\dashint_{0}^{p_{max}}\dd p''\frac{1}{q^2-p''^2}+i\pi\frac{f(q)}{2q}\\
	
	V_l(p, p')&=&t_l(p, p')-2\mu\int_0^{\infty}\dd p''\frac{p''^2V_l(p, p'')t_l(p'', p')-q^2V_l(p, q)t_l(q, p')}{q^2-p''^2}\\
	&&-2\mu q^2V_l(p, q)t_l(q, p')\frac{1}{2q}\ln\left(\frac{p_{max}+q}{p_{max}-q}\right)+i\pi\frac{2\mu q^2V_l(p, q)t_l(q, p')}{2q}\\
	
	V_l(p, p')&=&t_l(p, p')-2\mu\int_0^{\infty}\dd p''\frac{p''^2V_l(p, p'')t_l(p'', p')-q^2V_l(p, q)t_l(q, p')}{q^2-p''^2}\\
	&&-\mu qV_l(p, q)t_l(q, p')\ln\left(\frac{p_{max}+q}{p_{max}-q}\right)+i\pi\mu qV_l(p, q)t_l(q, p')\\
	
	V_{ij}&=&t_{ij}-2\mu\sum_{k=0}^{N-1}\omega_k\left(\frac{p_k^2V_{ik}t_{kj}}{q^2-p_k^2}-\frac{q^2V_{iN}t_{Nj}}{q^2-p_k^2}\right)-\mu qV_{iN}t_{Nj}\ln\left(\frac{p_{max}+q}{p_{max}-q}\right)+i\pi\mu qV_{iN}t_{Nj}\\
	
	&=&\sum_{k=0}^{N-1}\left(\delta_{ik}-\frac{2\mu p_k^2V_{ik}}{q^2-p_k^2}\omega_k\right)t_{kj}\\
	&&+\left(\delta_{iN}+\sum_{k=0}^{N-1}\frac{2\mu q^2V_{iN}}{q^2-p_k^2}\omega_k-\mu qV_{iN}\ln\left(\frac{p_{max}+q}{p_{max}-q}\right)+i\pi\mu qV_{iN}\right)t_{Nj}\\
	&=&\sum_{k=0}^{N}A_{ik}t_{kj}\\
	
	
	\end{longtable}
	
	This is exactly the matrix given on the sheet.
	
\subsection{Calculation of the potential}
	
	The potential was given in the lectures as \[
	V_l(p, p')=2\pi\int_{-1}^{1}\dd x P_l(x)\frac{1}{p^2+p'^2-2pp'x+\mu^2}
	\]
	
	We also discretize this integral using the Gauss-Legendre-approximation. 
	introduce s-wave counter term, limitations through cut-off value.
	
\subsection{Calculation of the T- and S-Matrix and the phase}
	
	For determining $t(q, q)=t_{NN}$, we use the following manipulation:

\begin{align*}
\sum_{k=0}^{N}A_{ik}t_{kj}&=V_{ij}\\
\sum_{k=0}^{N}A_{ik}t_{kN}&=V_{iN}\\
\end{align*}

Here, we can identify $A_{ik}$ as a matrix and $t_{kN}, V_{iN}$ as column vectors, with $V_{iN}$ being the last column of $V_{ij}$. We can solve this system of linear equations, and so determine $t_{kN}$ and from there $t_{NN}$. Then we can use the formula given on the sheet,\[
S(q)=1-2\pi i \mu q t(q)
\]

to calculate $S$. Then we can also determine $\delta(q)=\frac{1}{2}\arg(S(q))$.
	
\subsection{Calculation of the Crosssection}
	
\section{Implementation}

Our code is in the github-repo \url{https://github.com/christianegross/CompPhys\_2021}. The simulation itself is in Exercise7/src/main/main.c, the gnuplotscript is in Exercise7/report/plot.gp. 

GSL-functions used: Gauss-Legendre integration, Gaus-Legendre sf, Matrices, Vectors, BLAS and linalg for invertin matrix, complex numbers for matrix, t, s

\newpage
\listoffigures
\listoftables
\printbibliography
\end{document}
