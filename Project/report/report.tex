% ****** Start of file templateForReport.tex ******

% TeX'ing this file requires that you have all prerequisites
% for REVTeX 4.1 installed
%
% See the REVTeX 4 README file
% It also requires running BibTeX. The commands are as follows:
%
%  1)  latex templateForReport.tex
%  2)  bibtex templateForReport
%  3)  latex templateForReport.tex
%  4)  latex templateForReport.tex
%
\documentclass[%
 reprint,
%superscriptaddress,
%groupedaddress,
%unsortedaddress,
%runinaddress,
%frontmatterverbose,
%preprint,
%showpacs,preprintnumbers,
%nofootinbib,
%nobibnotes,
%bibnotes,
 amsmath,amssymb,
 aps,
%pra,
%prb,
%rmp,
%prstab,
%prstper,
%floatfix,
]{revtex4-1}

\usepackage{graphicx}% Include figure files
\usepackage{dcolumn}% Align table columns on decimal point
\usepackage{bm}% bold math
%\usepackage{hyperref}% add hypertext capabilities
%\usepackage[mathlines]{lineno}% Enable numbering of text and display math
%\linenumbers\relax % Commence numbering lines

%\usepackage[showframe,%Uncomment any one of the following lines to test
%%scale=0.7, marginratio={1:1, 2:3}, ignoreall,% default settings
%%text={7in,10in},centering,
%%margin=1.5in,
%%total={6.5in,8.75in}, top=1.2in, left=0.9in, includefoot,
%%height=10in,a5paper,hmargin={3cm,0.8in},
%]{geometry}

\begin{document}

\title{Yang-Mills on the Lattice}% Force line breaks with \\
%\thanks{A footnote to the article title}%

\author{Nico Dichter}
\author{Christiane Gro\ss{}}


\date{March 2021}% It is always \today, today,
             %  but any date may be explicitly specified

\begin{abstract}
	We measure the confinement that arises when simulating a Lattice with the Yang-Mills action, by approximating the lattice links with SU(2) matrices
%  An article usually includes an abstract, a concise summary of the work
%  covered at length in the main body of the article.
%  \begin{description}
%  \item[Usage]
%    Secondary publications and information retrieval purposes.
%  \item[Structure]
%    You may use the \texttt{description} environment to structure your abstract;
%    use the optional argument of the \verb+\item+ command to give the category of each item.
%  \end{description}
\end{abstract}
\maketitle

%\tableofcontents

\section{Introduction}

some random text

\section{Theoretical basis}

Some citation \citet{gsldoc_complex}

\section{Methods}

\section{Results}

\section{Discussion}

\section{Summary}

%complicated way of specifying file, but does not need absolute path and simply writing refs did not work
\bibliography{../report/refs}

\end{document}

