%!TeX spellcheck = en_GB
% Die erste (unkommentierte) Zeile im Dokument legt immer die
% Dokumentklasse fest
\documentclass{scrartcl} 

% Präambel:
% Einbinen von zusätzlichen Paketen. Falls für eine Datei keine Endung
% explizit angegeben wird, benutzt LaTeX '.tex'. Im Folgenden wird
% also die Datei 'edv_pakete.tex' eingebunden.
% Die erste Zeile im Dokument legt immer die Dokumentklasse fest
%\documentclass[notitlepage]{scrreprt}
    % Die wichtigsten Dokumentklassen:
    %   scrbook, scrreprt, scrartcl, beamer, standalone
    % Einige gängige Optionen für \documentclass:
    %   ngerman
    %   titlepage, notitlepage
    %   onecolumn, twocolumn
    %   oneside, twoside
    %wird in Hauptdatei festgelegt

% Präambel

% Einige KOMA-Script-Optionen
\KOMAoptions{fontsize=12pt,paper=a4}      %Schriftgröße, Papierformat
\KOMAoptions{DIV=11}                      % Parameter mit dem man den Seitenrand ändern kann
\KOMAoptions{listof=totoc}

% Hier werden einige Pakete eingebunden
\usepackage[utf8]{inputenc}               % Direkte Eingabe von ä usw. Input=Eingabe
\usepackage[T1]{fontenc}                  % Font Kodierung für die Ausgabe Font=Ausgabe
\usepackage[english]{babel}               % Verschiedenste sprach-spezifische Extras, ngerman für neue deutsche Rechtschreibung, auch UK oder US möglich
\usepackage[autostyle=true]{csquotes}     % Intelligente Anführungszeichen, arbeitet mit Babel zusammen
%

\usepackage{amsmath}%Mathedarstellung
\usepackage{commath}%Mathedarstellung
\usepackage{physics}%Physik-Symbole
%\usepackage{IEEEtrantools}%IEEEeqnarray
%
\usepackage{siunitx}   % Intelligentes Setzen von Zahlen und Einheiten
%\sisetup{locale = DE}  % Deutsch als locale für die Zahlen und Einheiten
%http://tex.stackexchange.com/questions/2291/how-do-i-change-the-enumerate-list-format-to-use-letters-instead-of-the-defaul

\usepackage{enumitem}%erlaubt u.A. die Aufzählung mit Buchstaben, gefunden auf http://tex.stackexchange.com/questions/2291/how-do-i-change-the-enumerate-list-format-to-use-letters-instead-of-the-defaul
%
\usepackage[varg]{txfonts}                % Schönere Schriftart, muss nach amsmath, damit keine Fehlermeldung kommt
\usepackage{graphicx} %einbinden von Figuren/Bildern
\graphicspath{{figs/}} % Stammverzeichnis der verwendeten Bilder, muss im selben Ordner wie Hauptdatei sein
%
\usepackage[backend=biber, style=numeric, sorting=none]{biblatex}
%Verwenden von \cite in \footnote: Bibliographie drucken lassen, mehrmals kompilieren
\usepackage{hyperref}%erzeugt klickbare Elemente
\usepackage[all]{hypcap}%hyperref-befehle springen zum oberen Rand des Bildes
% Zum Einbinden von Programmcode verwenden wir das listings-Paket
\usepackage{listings}

% Für Syntax-Highlighting:
\usepackage{xcolor}

\usepackage{longtable}

% Die folgenden listings-Einstellungen sind nötig, um
% deutsche Umlaute und die Tilde (~) in listings-Umgebungen
% verwenden zu können.
\lstset{
    basicstyle=\ttfamily,    
    literate={~} {$\sim$}{1} % set tilde as a literal
    {ö}{{\"o}}1
    {ä}{{\"a}}1
    {ü}{{\"u}}1
    {ß}{{\ss}}1
    {Ö}{{\"O}}1
    {Ä}{{\"A}}1
    {Ü}{{\"U}}1
}

% Farben für Code-Syntaxhighlighting und Weiteres festlegen:
\lstset{
    % Keine besondere Markierung für Leerzeichen in Codes
    showspaces=false,               
    showstringspaces=false,         
    % Farebn für Code-Kommentare und Schlüsselworte:
    commentstyle=\color{red},       % comment style
    keywordstyle=\color{blue},      % keyword style
    stringstyle=\color{orange},		% string style
    breaklines=true,
    numbers=left,                    % where to put the line-numbers; possible values are (none, left, right)
    numbersep=5pt,                   % how far the line-numbers are from the code
    stepnumber=5, 					%how often there are line numbers in code listings
    tabsize=4, 						%default tabsize set to 4 spaces
    %language=python,
    }
%gefunden auf https://en.wikibooks.org/wiki/LaTeX/Source_Code_Listings
%eigene Kommandos/Abürzungen
\newcommand{\tb}{\textbackslash}
\newcommand{\txt}{\texttt}
\newcommand{\umt}{u_{(i+i\%2)/2}^{(2a)}}
\newcommand{\utmt}{u_{(i-2+i\%2)/2}^{(2a)}}
\newcommand{\uti}{\tilde{u}_i^{(a)}}
\newcommand{\utio}{\tilde{u}_{i-1}^{(a)}}




% Verzeichnisse mit Abbildungen; kann gestrichen werden,
% falls Sie dies schon in edv_pakete.tex definiert haben:
%\graphicspath{{../report}}

\addbibresource{refs.bib} %Hinzufügen einer Literaturdatenbank aus dem angegebenen Verzeichnis

% Titel, Autor und Datum
\title{Computational Physics}
\subtitle{Exercise 5}
\date{\today}
\author{Christiane Groß, Nico Dichter}

% Jetzt startet das eigentliche Dokument
\begin{document}
	\maketitle
	
\section{Theory}
\subsection{The Gaussian Model}

We want to simulate the Gaussian model with the following hamiltonian that can also be discretized:
\[
H(u)=\int_{0}^{L}\mathrm{d}x\left(\partial{u(x)} \right)^2=\frac{1}{a}\sum_{i=1}^{N}\left(u_i-u_{i-1} \right)^2=H_a(u)
\]

For the discretization, we divide the length $L$ into $N$ pieces, each of which has length $a=L/N$. This give us $N+1$ points connecting the pieces to the end and to each other. 

We assume Dirichlet-boundary conditions, i.\! e.\! $u(0)=u(L)=0=u_0=u_N$.

Any observable can be calculated the usual way as: \[
\langle O\rangle=\frac{1}{Z}\prod_{i=1}^{N-1}\int\mathrm{d}u_i O(u)\exp(-\beta H(u)) \]
\[
Z=\prod_{i=1}^{N-1}\int\mathrm{d}u_i\exp(-\beta H(u)) 
\]

We set $\beta=1$ and ignore it.

For the multigrid algorithm, we also need a generalized hamiltonian that takes into accout an external field:
\[
H_a(u^{(a)})=\frac{1}{a}\sum_{i=1}^N
\left( u_i^{(a)}-u_{i-1}^{(a)}\right) ^2+a\sum_{i=1}^{N-1}\phi_i^{(a)}u_i^{(a)}
\]

When performing the simulation, we generate new states by selecting a random site, proposing a change $\Delta$ with $\Delta\sim\mathcal{U}[-\delta, \delta]$ and accepting it with probability $\min(1, \exp(H(u)-H(u')))$. By doing this $N-1$ times, we perform one sweep.

We are interested in measuring the average energy, $\langle H\rangle$, as well as the average magnetisation $\langle m \rangle$ and ist square $\langle m^2\rangle$, with $m=\frac{a}{L}\sum_{k=1}^{N-1}u_k$.

\subsection{Multigrid Algorithms}

We are interested in the behaviour of the system for small $a$ and large $N$, but simulating this is expensive in terms of computation time. The multigrid algorithm tries to solve this problem by doing most of the calculation at a coarser level and using these results for interpolating the finer levels.
We first do $\nu$ presweeps at the current level, then, if we aren't at the highest level yet, we do $\gamma$ multigrid sweeps at the next coarsest level, and then do $\nu$ postsweeps.

\paragraph{restrictions}
To be able to do sweeps at a coarser level, we need to determine the Hamiltonian and the lattice $u$ at that level. The derivation of the Hamiltonian is done in section~\ref{subsec:phi2a}, and we do the restriction of $u$ by taking every second grid point: $u_i^{(2a)}=u_{2i}^{(a)}$.

\paragraph{prolongations}
After the sweep at the coarser level, we need to add these results to the finer level. We do this by setting:\[
I^{(a)}_{(2a)}=\begin{cases}
u_{i/2}^{(2a)}& \text{i even}\\
\frac{1}{2}\left(u_{(i-1)/2}^{(2a)}u_{(i+1)/2}^{(2a)}\right) & \text{i odd}\\
\end{cases}
\]

and get our new field by setting $\tilde{u}^{(a)}+I_{2a}^au^{(2a)}$.

\section{Deliberations}

\subsection{analytical solutions}
We were not able to calculate the expactation values of our observables analytically.
%\begin{longtable}{>{$\displaystyle}r<{$}>{$\displaystyle}c<{$}>{$\displaystyle}l<{$}}
%	\alpha_k&=&\frac{2N}{a}\sin^2\left(\frac{k\pi}{2N}\right)\\
%	H_a(u)&=&\sum_{k=1}^{N-1}c_k^2\alpha_k\\
%	Z(a, N)&=&\prod_{k=1}^{N-1}\int\mathrm{d}c_k\exp(-H_a(u))\\
%	&=&\prod_{k=1}^{N-1}\int\mathrm{d}c_k\exp(-\alpha_kc_k^2)=\prod_{k=1}^{N-1}\frac{1}{2}\sqrt{\frac{\pi}{\alpha_k}}\\
%	
%	\langle E\rangle(a, N) &=&\frac{1}{Z}\prod_{k=1}^{N-1}\int\mathrm{d}c_kH_a(u)\exp(-H_a(u))\\
%	&=&\frac{1}{Z}\prod_{k=1}^{N-1}\int\mathrm{d}c_k\, c_k^2\alpha_k\exp(-H_a(u))\\
%	&=&\frac{1}{Z}\prod_{k=1}^{N-1} \alpha_k \frac{1}{2}\sqrt{\frac{\pi}{\alpha_k^3}}\\
%	&=&\frac{1}{Z} Z\prod_{k=1}^{N-1}\frac{\alpha_k}{\alpha_k}=1\\
%	
%	m&=&\frac{a}{L}\sum_{l=1}^{N-1}u_l=\frac{a}{L}\sum_{l=1}^{N-1}\sum_{k=1}^{N-1}c_k\sin\left(\frac{k\pi l}{N} \right) \\
%	&=&\frac{a}{L}\sum_{k=1}^{N-1}c_k\frac{1}{2}(1-(-1)^k)cot\left( \frac{k\pi}{2N}\right) \\
%	&\Rightarrow & m\text{ is an odd function in } c_k\\
%	\langle m\rangle &=&\frac{1}{Z}\prod_{k=1}^{N-1}\int\mathrm{d}c_k\underbrace{m}_{\text{odd}}\underbrace{\exp(-H_a(u))}_{\text{even}}=0\\
%	
%	
%	\langle m^2 \rangle &=&\frac{1}{Z}\prod_{k=1}^{N-1}\int\mathrm{d}c_k \frac{a^2}{L^2}\frac{1}{4}(1-(-1)^k)^2\cot^2\left(\frac{k\pi}{2N}\right)c_k^2\exp(-H_a(u))\\
%	&=&\frac{1}{Z}\prod_{k=1}^{N-1}\frac{a^2}{L^2} \frac{1}{4}(1-(-1)^k)^2\cot^2\left(\frac{k\pi}{2N}\right)\frac{1}{2}\sqrt{\frac{\pi}{\alpha_k^3}}\\
%	&=&\prod_{k=1}^{N-1}\frac{a^2}{L^2} \frac{a}{8N}(1-(-1)^k)^2\frac{\cot^2\left(\frac{k\pi}{2N}\right)}{\sin^2\left(\frac{k\pi}{2N}\right)}\\
%	&=&\prod_{k=1}^{N-1}\frac{a}{8N^3} (1-(-1)^k)^2\frac{\cos^2\left(\frac{k\pi}{2N}\right)}{\sin^4\left(\frac{k\pi}{2N}\right)}\\
%%	&&
%\end{longtable}

\subsection{Explicit form of $\phi^{(2a)}$}
$\%$ is the modulus operator.
\label{subsec:phi2a}

\begin{longtable}{>{$\displaystyle}r<{$}>{$\displaystyle}c<{$}>{$\displaystyle}l<{$}}
I_{2a}^a\left( u_i^{(a)}-u_{i-1}^{(a)}\right)  &=&u_{i/2}^{(2a)}-\frac{1}{2}\left(u_{i/2}^{(2a)}+u_{(i-2)/2}^{(2a)} \right) \text{  for i even} \\
&=&\frac{1}{2}\left(u_{i/2}^{(2a)}-u_{(i-2)/2}^{(2a)}\right) \\

I_{2a}^a\left( u_i^{(a)}-u_{i-1}^{(a)}\right)   &=&\frac{1}{2}\left(u_{(i-1)/2}^{(2a)}+u_{(i+1)/2}^{(2a)}-u_{(i-1)/2}^{(2a)} \right)\text{  for i odd} \\
&=&\frac{1}{2}\left(u_{(i+1)/2}^{(2a)}-u_{(i+1)/2}^{(2a)}\right) \\

\Leftrightarrow I_{2a}^a\left( u_i^{(a)}-u_{i-1}^{(a)}\right)&=&
\frac{1}{2}\left(u_{(i+i\%2)/2}^{(2a)}-u_{(i-2+i\%2)/2}^{(2a)}\right) \\


\end{longtable}


\begin{longtable}{>{$\displaystyle}r<{$}>{$\displaystyle}c<{$}>{$\displaystyle}l<{$}}

H_a(u^{(a)})&=&\frac{1}{a}\sum_{i=1}^N
\left( u_i^{(a)}-u_{i-1}^{(a)}\right) ^2+a\sum_{i=1}^{N-1}\phi_i^{(a)}u_i^{(a)}\\
&=&H_a\left( \tilde{u}^{(a)}+I_{2a}^au^{(2a)}\right) \\

&=&\frac{1}{a}\sum_{i=1}^N\left( \tilde{u}_i^{(a)}-\tilde{u}_{i-1}^{(a)}+\frac{1}{2}u_{(i+i\%2)/2}^{(2a)}-\frac{1}{2}u_{(i-2+i\%2)/2}^{(2a)}\right) ^2+a\sum_{i=1}^{N-1}\left( \phi_i^{(a)}\tilde{u}_i^{(a)}+\phi_i^{(a)}\left( I_{2a}^au^{(2a)}\right)_i\right) \\

&&\\

&=&\frac{1}{a}\sum_{i=1}^N\left((\uti)^2+(\utio)^2-2\uti\utio+\frac{1}{4}(\umt)^2+\frac{1}{4}(\utmt)^2-\frac{1}{2}\utmt\umt\right) \\
&+&\frac{1}{a}\sum_{i=1}^N\left(\umt\uti-\umt\utio-\utmt\uti+\utmt\utio\right)\\
&+&a\sum_{i=1}^{N-1}\left( \phi_i^{(a)}\tilde{u}_i^{(a)}+\phi_i^{(a)}\left( I_{2a}^au^{(2a)}\right)_i\right)\\

&&\\

&=&\frac{1}{a}\sum_{i=1}^N
\left( \uti-\utio\right) ^2+a\sum_{i=1}^{N-1}\phi_i^{(a)}\uti+
\frac{1}{4a}\sum_{i=1}^N\left(\umt-\utmt\right)^2\\

&+&\frac{1}{a}\sum_{i=1}^N \left(\umt\left(\uti-\utio \right) +\utmt\left(\utio-\uti \right)  \right)+a\sum_{i=1}^{N-1}\phi_i^{(a)}\left( I_{2a}^au^{(2a)}\right)_i\\

&&\\

&=&H_a\left(  \tilde{u}^{(a)}\right) + \frac{1}{2a}\sum_{i=1}^{N/2} \left( u_i^{(2a)}-u_{i-1}^{(2a)}\right)^2\\
&+&\frac{1}{a}\sum_{i=1}^{N/2}
\left( u_{i}^{(2a)}\left( \tilde{u}_{2i}^{(a)}\underbrace{-\tilde{u}_{2i-1}^{(a)}+\tilde{u}_{2i-1}^{(a)}}_{=0}-\tilde{u}_{2i-2}^{(a)}\right) 
+u_{i-1}^{(2a)}\left(\tilde{u}_{2i-2}^{(a)}\underbrace{-\tilde{u}_{2i-1}^{(a)}+ \tilde{u}_{2i-1}^{(a)}}_{=0}-\tilde{u}_{2i}^{(a)}\right) \right)\\
&+&a\sum_{i=1}^{N/2-1}
\left( \phi_{2i}^{(a)}+\frac{1}{2}\phi_{2i+1}^{(a)}+\frac{1}{2}\phi_{2i-1}^{(a)}\right) u_i^{(2a)} \\

&&\\


&=&H_a\left(  \tilde{u}^{(a)}\right) + \frac{1}{2a}\sum_{i=1}^{N/2} \left( u_i^{(2a)}-u_{i-1}^{(2a)}\right)^2\\
&+&a\sum_{i=1}^{N/2-1}
u_{i}^{(2a)}\frac{1}{a^2}\left(\tilde{u}_{2i}^{(a)}-\tilde{u}_{2i-2}^{(a)}-\tilde{u}_{2i+2}^{(a)}+\tilde{u}_{2i}^{(a)}  \right)
+\left( \phi_{2i}^{(a)}+\frac{1}{2}\phi_{2i+1}^{(a)}+\frac{1}{2}\phi_{2i-1}^{(a)}\right) u_i^{(2a)} \\

&&\\

%&=&H_a\left( \tilde{u}^{(a)}+I_{2a}^au^{(2a)}\right) \\\\


&=&H_a\left(  \tilde{u}^{(a)}\right) + \frac{1}{2a}\sum_{i=1}^{N/2} \left( u_i^{(2a)}-u_{i-1}^{(2a)}\right)^2 +2a\sum_{i=1}^{N/2-1}\phi_i^{(2a)}u_i^{(2a)}\\

&=&H_a\left( \tilde{u}^{(a)}\right) +H_{2a}\left( u^{(2a)}\right) \\

\Leftrightarrow \phi_i^{(2a)}&=&\frac{1}{2a^2}\left(2\tilde{u}_{2i}^{(a)}-\tilde{u}_{2i-2}^{(a)}-\tilde{u}_{2i+2}^{(a)}\right)+\frac{1}{2}\left( \phi_{2i}^{(a)}+\frac{1}{2}\phi_{2i+1}^{(a)}+\frac{1}{2}\phi_{2i-1}^{(a)}\right)\\

\end{longtable}

\subsection{Accept/Reject step}

During the sweep, we propose $u_l\to u_l+\Delta$ for a randomly chosen $l$. For the accept/reject step, we need the difference in the Hamiltonians $H(u)-H(u+\Delta)$ of the two configurations, but we do not need to calculate the entire Hamiltonian:

\[\begin{array}{>{\displaystyle}r>{\displaystyle}c>{\displaystyle}l}
H(u+\Delta)-H(u)&=&\frac{1}{a}\left( -\left(u_l-u_{l-1} \right)^2-\left(u_{l+1}-u_{l}  \right)^2+\left(u_l+\Delta-u_{l-1}  \right)^2+\left( u_{l+1}-u_{l}-\Delta \right)^2\right) \\&+&a\left( -\phi_lu_l+\phi_lu_l+\phi_l\Delta\right) \\

%&=&\frac{1}{a}\left( u_l^2+u_{l-1}^2-2u_lu_{l-1}+u_{l+1}^2+u_l^2-2u_{l+1}u_l\\
%&-&\left(u_l^2+2u_l\Delta-2u_lu_{l-1}+\Delta^2-2\Delta u_{l-1}+u_{l-1}^2+u_{l+1}^2-2u_{l+1}u_l-2u_{l+1}\Delta+u_l^2+2u_l\Delta+\Delta^2 \right)\right) -a\phi_l\Delta \\


&=&\frac{1}{a}\left(+2u_l\Delta+\Delta^2-2\Delta u_{l-1}-2u_{l+1}\Delta+2u_l\Delta+\Delta^2 \right) +a\phi_l\Delta \\

&=&\frac{2\Delta}{a}\left(2u_l-u_{l-1}-u_{l+1}+\Delta \right)+ a\phi_l\Delta
%\sum_{i=1}^N
%\left( u_i^{(a)}-u_{i-1}^{(a)}\right) ^2+a\sum_{i=1}^{N-1}\phi_i^{(a)}u_i^{(a)}\\
%&=&H_a\left( \tilde{u}^{(a)}+I_{2a}^au^{(2a)}\right) \\

\end{array}\]

\section{Implementation}

Our code is in the github-repo \url{https://github.com/christianegross/CompPhys\_2021}. The simulation itself is in Exercise5/src/main/main.c, the gnuplotscript is in Exercise5/report/plot.gp. 

\section{Results}
\subsection{Simulation at the finest level}
compare result of m, $m^2$, e to expectation

\subsection{Simulation of the Multigrid-algorithm}
\paragraph{$\gamma=1$}
plot autocorrelation
compare result of m, $m^2$, e to expectation
\paragraph{$\gamma=2$}
plot autocorrelation
compare result of m, $m^2$, e to expectation

\newpage	
\listoffigures
\printbibliography
\end{document}