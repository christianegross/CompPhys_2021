% Die erste Zeile im Dokument legt immer die Dokumentklasse fest
%\documentclass[notitlepage]{scrreprt}
    % Die wichtigsten Dokumentklassen:
    %   scrbook, scrreprt, scrartcl, beamer, standalone
    % Einige gängige Optionen für \documentclass:
    %   ngerman
    %   titlepage, notitlepage
    %   onecolumn, twocolumn
    %   oneside, twoside
    %wird in Hauptdatei festgelegt

% Präambel

% Einige KOMA-Script-Optionen
\KOMAoptions{fontsize=12pt,paper=a4}      %Schriftgröße, Papierformat
\KOMAoptions{DIV=11}                      % Parameter mit dem man den Seitenrand ändern kann
\KOMAoptions{listof=totoc}

% Hier werden einige Pakete eingebunden
\usepackage[utf8]{inputenc}               % Direkte Eingabe von ä usw. Input=Eingabe
\usepackage[T1]{fontenc}                  % Font Kodierung für die Ausgabe Font=Ausgabe
\usepackage[english]{babel}               % Verschiedenste sprach-spezifische Extras, ngerman für neue deutsche Rechtschreibung, auch UK oder US möglich
\usepackage[autostyle=true]{csquotes}     % Intelligente Anführungszeichen, arbeitet mit Babel zusammen
%

\usepackage{amsmath}%Mathedarstellung
\usepackage{commath}%Mathedarstellung
%\usepackage{IEEEtrantools}%IEEEeqnarray
%
\usepackage{siunitx}   % Intelligentes Setzen von Zahlen und Einheiten
%\sisetup{locale = DE}  % Deutsch als locale für die Zahlen und Einheiten
%http://tex.stackexchange.com/questions/2291/how-do-i-change-the-enumerate-list-format-to-use-letters-instead-of-the-defaul

\usepackage{enumitem}%erlaubt u.A. die Aufzählung mit Buchstaben, gefunden auf http://tex.stackexchange.com/questions/2291/how-do-i-change-the-enumerate-list-format-to-use-letters-instead-of-the-defaul
%
\usepackage[varg]{txfonts}                % Schönere Schriftart, muss nach amsmath, damit keine Fehlermeldung kommt
\usepackage{graphicx} %einbinden von Figuren/Bildern
\graphicspath{{figs/}} % Stammverzeichnis der verwendeten Bilder, muss im selben Ordner wie Hauptdatei sein
%
\usepackage[backend=biber, style=numeric]{biblatex}
%Verwenden von \cite in \footnote: Bibliographie drucken lassen, mehrmals kompilieren
\usepackage{hyperref}%erzeugt klickbare Elemente
\usepackage[all]{hypcap}%hyperref-befehle springen zum oberen Rand des Bildes
% Zum Einbinden von Programmcode verwenden wir das listings-Paket
\usepackage{listings}

% Für Syntax-Highlighting:
\usepackage{xcolor}

% Die folgenden listings-Einstellungen sind nötig, um
% deutsche Umlaute und die Tilde (~) in listings-Umgebungen
% verwenden zu können.
\lstset{
    basicstyle=\ttfamily,    
    literate={~} {$\sim$}{1} % set tilde as a literal
    {ö}{{\"o}}1
    {ä}{{\"a}}1
    {ü}{{\"u}}1
    {ß}{{\ss}}1
    {Ö}{{\"O}}1
    {Ä}{{\"A}}1
    {Ü}{{\"U}}1
}

% Farben für Code-Syntaxhighlighting und Weiteres festlegen:
\lstset{
    % Keine besondere Markierung für Leerzeichen in Codes
    showspaces=false,               
    showstringspaces=false,         
    % Farebn für Code-Kommentare und Schlüsselworte:
    commentstyle=\color{red},       % comment style
    keywordstyle=\color{blue},      % keyword style
    stringstyle=\color{orange},		% string style
    breaklines=true,
    numbers=left,                    % where to put the line-numbers; possible values are (none, left, right)
    numbersep=5pt,                   % how far the line-numbers are from the code
    stepnumber=5, 					%how often there are line numbers in code listings
    tabsize=4, 						%default tabsize set to 4 spaces
    %language=python,
    }
%gefunden auf https://en.wikibooks.org/wiki/LaTeX/Source_Code_Listings
%eigene Kommandos/Abürzungen
\newcommand{\tb}{\textbackslash}
\newcommand{\txt}{\texttt}

