% !TeX spellcheck = en_UK
% Die erste (unkommentierte) Zeile im Dokument legt immer die
% Dokumentklasse fest
\documentclass{scrartcl} 

% Präambel:
% Einbinen von zusätzlichen Paketen. Falls für eine Datei keine Endung
% explizit angegeben wird, benutzt LaTeX '.tex'. Im Folgenden wird
% also die Datei 'edv_pakete.tex' eingebunden.
\input{edv_pakete}


% Verzeichnisse mit Abbildungen; kann gestrichen werden,
% falls Sie dies schon in edv_pakete.tex definiert haben:
%\graphicspath{{Blatt2/}}

\addbibresource{refs.bib} %Hinzufügen einer Literaturdatenbank aus dem angegebenen Verzeichnis

% Titel, Autor und Datum
\title{Computational Physics}
\subtitle{Exercise 1}
\date{\today}
\author{Christiane Groß, Nico Dichter}

% Jetzt startet das eigentliche Dokument
\begin{document}
	\maketitle

\section{Ising-Model in 1 dimension}
copy definitions of H, Z, m from sheet

\section{Preliminary deliberations}

Before simulating the model, we first have to think about several aspects
	
\subsection{What is the physical meaning of J?}

strength of connection between spins, also called exchange energy~\cite{binderheermann}, bigger J=stronger magnet, J>0: ferromagnetic, J<0 antiferromagnetic

\subsection{What are periodic boundary conditions?}
In 1D, every state has to have 2 neighbours. What about spins at end of lattice? Neighbour=Spin at other end of lattice. Implemented by using modulo operator on indices.

\subsection{What are the relevant dimensionless ratios?}

J/T, h/T, as seen in arguments of cosh/sinh and exponents of exp in Z. Need to vary temperature to get results.

\subsection{What are we wxpecting for the magnetization?}
calculate -T/N dlogZ/dh, maybe plot?
	
\subsection{How to determine the error on the measurements?}
L=number of random configurations sampled

\[\sigma^2(m)=\langle(\langle m\rangle-m)^2\rangle=\frac{1}{Z}\sum_{i=0}^{n=L}(\langle m\rangle-m_i)^2\exp(\frac{-H(s_i)}{T})\]

\section{Simulation}
Using gsl\_rng generator with Mersenne-Twister\cite{gsldoc}

for range of h or N
for range of temperatures
(for range of number of configs:
generate random state
calculate m, H, weight exp(-h/T), write in txt-file, add weight to Z)
calculate Z
read file to calculate mean(m)
read file again to calculate variance(m)
write maen/variance of m in file

plot with gnuplot

\section{Results}
\subsection{Varying h}
plot
\subsection{Varying N}
plot
\subsection{thermodynamical limit}

	
%	\begin{table}[htbp]
%		\centering
%		\begin{tabular}{l||l|l}
%			&Float&Double\\
%			\hline
%			&&\\
%			$h_{opt}$&$2.630\cdot10^{-3}$&$3.236\cdot10^{-6}$\\
%			$\delta_M$&$7.429\cdot10^{-9}$&$1.383\cdot10^{-17}$\\
%			$k$&$6.918\cdot10^{-5}$&$1.514\cdot10^{-8}$\\
%		\end{tabular} 
%		\caption{Ergebnisse für die Drei-Punkt-Ableitung} 
%		\label{tab:Drei-Punkt}
%	\end{table}
%	
	
\newpage	
\listoffigures
\printbibliography
\end{document}