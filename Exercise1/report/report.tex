% !TeX spellcheck = en_UK
% Die erste (unkommentierte) Zeile im Dokument legt immer die
% Dokumentklasse fest
\documentclass{scrartcl} 

% Präambel:
% Einbinen von zusätzlichen Paketen. Falls für eine Datei keine Endung
% explizit angegeben wird, benutzt LaTeX '.tex'. Im Folgenden wird
% also die Datei 'edv_pakete.tex' eingebunden.
% Die erste Zeile im Dokument legt immer die Dokumentklasse fest
%\documentclass[notitlepage]{scrreprt}
    % Die wichtigsten Dokumentklassen:
    %   scrbook, scrreprt, scrartcl, beamer, standalone
    % Einige gängige Optionen für \documentclass:
    %   ngerman
    %   titlepage, notitlepage
    %   onecolumn, twocolumn
    %   oneside, twoside
    %wird in Hauptdatei festgelegt

% Präambel

% Einige KOMA-Script-Optionen
\KOMAoptions{fontsize=12pt,paper=a4}      %Schriftgröße, Papierformat
\KOMAoptions{DIV=11}                      % Parameter mit dem man den Seitenrand ändern kann
\KOMAoptions{listof=totoc}

% Hier werden einige Pakete eingebunden
\usepackage[utf8]{inputenc}               % Direkte Eingabe von ä usw. Input=Eingabe
\usepackage[T1]{fontenc}                  % Font Kodierung für die Ausgabe Font=Ausgabe
\usepackage[english]{babel}               % Verschiedenste sprach-spezifische Extras, ngerman für neue deutsche Rechtschreibung, auch UK oder US möglich
\usepackage[autostyle=true]{csquotes}     % Intelligente Anführungszeichen, arbeitet mit Babel zusammen
%

\usepackage{amsmath}%Mathedarstellung
\usepackage{commath}%Mathedarstellung
\usepackage{physics}%Physik-Symbole
%\usepackage{IEEEtrantools}%IEEEeqnarray
%
\usepackage{siunitx}   % Intelligentes Setzen von Zahlen und Einheiten
%\sisetup{locale = DE}  % Deutsch als locale für die Zahlen und Einheiten
%http://tex.stackexchange.com/questions/2291/how-do-i-change-the-enumerate-list-format-to-use-letters-instead-of-the-defaul

\usepackage{enumitem}%erlaubt u.A. die Aufzählung mit Buchstaben, gefunden auf http://tex.stackexchange.com/questions/2291/how-do-i-change-the-enumerate-list-format-to-use-letters-instead-of-the-defaul
%
\usepackage[varg]{txfonts}                % Schönere Schriftart, muss nach amsmath, damit keine Fehlermeldung kommt
\usepackage{graphicx} %einbinden von Figuren/Bildern
\graphicspath{{figs/}} % Stammverzeichnis der verwendeten Bilder, muss im selben Ordner wie Hauptdatei sein
%
\usepackage[backend=biber, style=numeric, sorting=none]{biblatex}
%Verwenden von \cite in \footnote: Bibliographie drucken lassen, mehrmals kompilieren
\usepackage{hyperref}%erzeugt klickbare Elemente
\usepackage[all]{hypcap}%hyperref-befehle springen zum oberen Rand des Bildes
% Zum Einbinden von Programmcode verwenden wir das listings-Paket
\usepackage{listings}

% Für Syntax-Highlighting:
\usepackage{xcolor}

\usepackage{longtable}

% Die folgenden listings-Einstellungen sind nötig, um
% deutsche Umlaute und die Tilde (~) in listings-Umgebungen
% verwenden zu können.
\lstset{
    basicstyle=\ttfamily,    
    literate={~} {$\sim$}{1} % set tilde as a literal
    {ö}{{\"o}}1
    {ä}{{\"a}}1
    {ü}{{\"u}}1
    {ß}{{\ss}}1
    {Ö}{{\"O}}1
    {Ä}{{\"A}}1
    {Ü}{{\"U}}1
}

% Farben für Code-Syntaxhighlighting und Weiteres festlegen:
\lstset{
    % Keine besondere Markierung für Leerzeichen in Codes
    showspaces=false,               
    showstringspaces=false,         
    % Farebn für Code-Kommentare und Schlüsselworte:
    commentstyle=\color{red},       % comment style
    keywordstyle=\color{blue},      % keyword style
    stringstyle=\color{orange},		% string style
    breaklines=true,
    numbers=left,                    % where to put the line-numbers; possible values are (none, left, right)
    numbersep=5pt,                   % how far the line-numbers are from the code
    stepnumber=5, 					%how often there are line numbers in code listings
    tabsize=4, 						%default tabsize set to 4 spaces
    %language=python,
    }
%gefunden auf https://en.wikibooks.org/wiki/LaTeX/Source_Code_Listings
%eigene Kommandos/Abürzungen
\newcommand{\tb}{\textbackslash}
\newcommand{\txt}{\texttt}
\newcommand{\umt}{u_{(i+i\%2)/2}^{(2a)}}
\newcommand{\utmt}{u_{(i-2+i\%2)/2}^{(2a)}}
\newcommand{\uti}{\tilde{u}_i^{(a)}}
\newcommand{\utio}{\tilde{u}_{i-1}^{(a)}}




% Verzeichnisse mit Abbildungen; kann gestrichen werden,
% falls Sie dies schon in edv_pakete.tex definiert haben:
%\graphicspath{{Blatt2/}}

\addbibresource{refs.bib} %Hinzufügen einer Literaturdatenbank aus dem angegebenen Verzeichnis

% Titel, Autor und Datum
\title{Computational Physics}
\subtitle{Exercise 1}
\date{\today}
\author{Christiane Groß, Nico Dichter}

% Jetzt startet das eigentliche Dokument
\begin{document}
	\maketitle

\section{Ising-Model in 1 dimension}
copy definitions of H, Z, m from sheet

\section{Preliminary deliberations}

Before simulating the model, we first have to think about several aspects
	
\subsection{What is the physical meaning of J?}

strength of connection between spins, also called exchange energy~\cite{binderheermann}, bigger J=stronger magnet, J>0: ferromagnetic, J<0 antiferromagnetic

\subsection{What are periodic boundary conditions?}
In 1D, every state has to have 2 neighbours. What about spins at end of lattice? Neighbour=Spin at other end of lattice. Implemented by using modulo operator on indices.

\subsection{What are the relevant dimensionless ratios?}

J/T, h/T, as seen in arguments of cosh/sinh and exponents of exp in Z. Need to vary temperature to get results.

\subsection{What are we wxpecting for the magnetization?}
calculate -T/N dlogZ/dh, maybe plot?
	
\subsection{How to determine the error on the measurements?}
L=number of random configurations sampled

\[\sigma^2(m)=\langle(\langle m\rangle-m)^2\rangle=\frac{1}{Z}\sum_{i=0}^{n=L}(\langle m\rangle-m_i)^2\exp(\frac{-H(s_i)}{T})\]

\section{Simulation}
Using gsl\_rng generator with Mersenne-Twister\cite{gsldoc}

for range of h or N
for range of temperatures
(for range of number of configs:
generate random state
calculate m, H, weight exp(-h/T), write in txt-file, add weight to Z)
calculate Z
read file to calculate mean(m)
read file again to calculate variance(m)
write maen/variance of m in file

plot with gnuplot

\section{Results}
\subsection{Varying h}
plot
\subsection{Varying N}
plot
\subsection{thermodynamical limit}

	
%	\begin{table}[htbp]
%		\centering
%		\begin{tabular}{l||l|l}
%			&Float&Double\\
%			\hline
%			&&\\
%			$h_{opt}$&$2.630\cdot10^{-3}$&$3.236\cdot10^{-6}$\\
%			$\delta_M$&$7.429\cdot10^{-9}$&$1.383\cdot10^{-17}$\\
%			$k$&$6.918\cdot10^{-5}$&$1.514\cdot10^{-8}$\\
%		\end{tabular} 
%		\caption{Ergebnisse für die Drei-Punkt-Ableitung} 
%		\label{tab:Drei-Punkt}
%	\end{table}
%	
	
\newpage	
\listoffigures
\printbibliography
\end{document}